\documentclass{article}
\usepackage[utf8]{inputenc}
\usepackage{authblk}
\usepackage{graphicx}
\usepackage[table]{xcolor}



\title{Draft Report Modelleren-B}
\author{Bart Stolk \and Dyami Silvério \and Wessel van Sommeren \and Severijn Wiechers}
\date{03 June 2021}

\begin{document}

\maketitle

\newpage
\section*{Summary}
\addcontentsline{toc}{section}{Summary}



\newpage
\tableofcontents



\newpage
\section{Introduction}

% Reducing costs is a valuable tool for any package delivery cooperation to stay relevant and make profits. This report will try to show how it is possible to create an algorithm that gives an optimum case of hubs such that the total cost is minimized % we moeten de hoofdvraag nog even correct opstellen.

Reducing costs is a valuable tool for any package delivery cooperation to stay relevant and make profits. Hereby, algorithms are a useful and necessary tool to optimize a given data set. In this case the goal of the report is to find an algorithm to minimize the cost of a package delivery company. Hereby a data set is used where the different transfer, collection and distribution costs are given per city. The objective of the algorithm is to find the cities where hubs should be build such that the total cost of sending and receiving packages can be brought down to a minimum. An important restraint in this matter is that packages can only move between hubs and cities and in particular not directly between cities themselves. 

Since this type of optimization problem is a common practice for companies, there already exists an ILP solver which can give the exact hubs for which have the lowest total cost, However, it has shown that these solvers can take up and lot op time and processing power. Therefore it is preferable to get an idea for which routes and hubs will give the best, or at least a proper solution to the problem. This will be done by creating different intuitive algorithms which will, if done correctly, give similar results using different methods.

In the first chapter the intuitive algorithms will be explained along with the function which is used to calculate the total costs of choosing a specific combination of hubs. Thereafter, the chapter will set forth the input and result of the ILP Solver. The section in chapter 4 will focus on a reflection between the result of the intuitive algorithm and the solution of the ILP solver. In the following chapter multiple expansions of the problem and intuitive algorithm will be given such that a better result is obtained. Finally, the last chapter will contain a conclusion of the report.






\newpage
\section{Intuitive Algorithm}
As mentioned in the introduction, it can be useful to use an intuitive algorithm in stead of a direct solver for the problem. With a self made script it is possible to get a feel for the cities in which building a hub can have a major impact on reducing the cost of the project. 
Before a code can be written that tries to optimize the building of hubs in cities, it is necessary to have a solid view of the costs that a specific combination of hubs will bring forth. Therefore a cost-function is written that will give a total cost as output with a combination of hubs as input. 



    \subsection{Calculating the cost for multiple hubs}
        To calculate the cost of using specific hubs we first choose a way to determine connections from hubs to cities. The most intuitive way to determine those connections was to link each city with the hub that was cheapest to collect and distribute to. Intuitively this made sense as it probably the closest. If we take for example the Netherlands this would mean that a packet send from Groningen would be picked up by a hub close to Groningen and not by a hub in for example Maastricht. So the first step in the cost function algorithm was to go trough the list of cities and determine which hub from the list of inputted hubs was cheapest to collect from for each city. When this connection is determined the sum of the list multiplied by the collection factor is added to the cost and the the amount of packages are added to the packages of the hub city. This connection between the city and collection hub was stored in memory as it would be used later to determine transfers and delivery. As soon as this first step is complete we end up with a Dataframe with as columns the hub cities and rows the packages that need to go to each city. The next step is to go row by row and look up what connection that city had with which hub and transfer all the packages of that row to the correct hub again multiplying the amount times the correct cost associated with transfering from hub to hub. Then the total of that row at the correct hub will be delivered to the city and thus multiplying the total number of packages by the cost to deliver from that hub to the city and the delivery factor. Once that is done for all rows, all packages will be delivered. Now we only need to add the cost of building the hubs and we have the total cost of using specific hubs. 
        
        
    \subsection{The different algorithms used}
    
    After obtaining a function which calculates the cost of assigning hubs to cities, an intuitive algorithm can be made which tries to give an approximation of the combination of hubs that will result in the lowest possible amount of money that the company will have to spend when moving packages. 
    Since the intuitive algorithms are inherently less accurate than a standardized solver, the different aspects of moving packages are split up and solved using different types of algorithms. Hereby, the goal is to minimize the error that the functions have by eventually combining multiple intuitive algorithms such that a minimum cost can be achieved. The total process is divided into the collection from the cities to the hubs, followed by the transfer between hubs themselves, and as last the distribution of packages from the hubs to the desired cities. 
    
\subsection{Collection of packages}    

In this process the cost function will only give an output of the cost that is formed by combining the cost for collection packages.
To optimize the costs a python script is made such that, when a city is given to be a hub, the code calculates the cost for this specific hub and the cost if the city next in line would be a hub as well. If the total cost drops by adding city as a hub, this city will be added to list of cities that are hubs. If the cost rises, the city will be discarded from the process. If this process is repeated with each time a different starting, a list will appear of possible combination of hubs and there costs. To illustrate this process the outputs will be stated below if it were to start with city number 3 as a hub.

\begin{table}[h!]
\begin{center}
 \begin{tabular}{||c |c||} 
 \hline
 Cost of the combination(€ x 10^3) & combination on cities as hubs \\ [0.5ex] 
 \hline\hline
 169.3 & [3,1]  \\ 
 \hline
 169.8 & [3,2]  \\
 \hline
 169.8 & [3,4]  \\
 \hline
 127.2 & [3,5]  \\ 
 \hline
 132.0 & [3,5,6]  \\
  \hline
 127.4 & [3,5,7]  \\
 \hline
 128.5 & [3,5,8]  \\
 \hline
 124.8 & [3,5,9]  \\
  \hline
 118.7 & [3,5,9,10]  \\
 [1ex] 
 \hline
\end{tabular}
\end{center}
\caption{Exampled of the algorithm which adds cities one by one as hubs based on the cost}
\label{Collection Algorithm}
\end{table}

Using this method, the algorithm can give the most frequently used cities as hubs. To illustrate the difference in accuracy between the different models the table below will give the amount of times cities are used as hubs in each different part.% Herlezen

\begin{table}
\begin{tabular}{|p{3cm}|p{3cm}|p{3cm}|}
\hline
\multicolumn{3}{|c|}{Amount of times cities are used as hubs} \\
\hline
City used as hub & Using only collection cost & Using total cost\\
\hline
1 &  8 & 5 \\
\hline
2 & 4   & 11 \\
\hline
3& 7 & 9\\
\hline
4    & 1 & 1 \\
\hline
5 & 8 & 9\\
\hline
6 & 1 & 8   \\
\hline
7 & 2& 3 \\

\hline
8 & 2 & 2 \\
\hline
9 &  7& 10 \\
\hline
10 & 11 & 11 \\
\hline
11 & 1& 2 \\
\hline
12 & 1 & 1 \\
\hline
13 & 2& 2 \\
\hline
14 & 14& 14 \\
\hline
\end{tabular}
\caption{Table of the amount of times each city is used as a hub in every combination}
\label{amount of times hubs are used in collection} 
\end{table}

Using this table a conclusion can be drawn which hubs could be used in the optimum configuration. However, a big part of cost reduction is the combination of specific hubs. Consider city number 6 for example, here it shows that making this city a hub is not a valuable if the algorithm only takes the collection cost in account, but rises significantly if all cost are calculated. 

When the algorithm is used for different input combinations based on ~\ref{amount of times hubs are used in collection} the outputs and results show different possible answers with supposedly relative low cost. Hereby the algorithm will add a city as hub if this results in a lower cost by the principle given in the beginning of the section. 

\begin{table}
\begin{tabular}{ |p{3cm}|p{3cm}|p{4cm}|}
\hline
\multicolumn{3}{|c|}{Cost and combination of hubs with certain input} \\
\hline
Input into the algorithm & Output -  collection Cost (€ x 10^3) & Output - total Cost (€ x 10^3)\\
\hline
[14] &  92.6 | [14, 1, 3] & 166.4 | [14, 1, 2, 3, 5, 7]  \\
\hline
[10] & 90.0 | [10, 2, 14]   & 153.0 | [10, 2, 14, 8] \\
\hline
[6, 10]& 93.9 | [6, 10, 14] & 150.4 |  [6, 10, 14]\\
\hline
[6, 14] &  82.9 | [6, 14]  &  144.5 | [6, 14, 8] \\
\hline
[9, 14] &  100.7| [9, 14, 1, 3] & 160.4 |  [9, 14, 2, 3, 7] \\
\hline
[2, 14] & 87.6 | [2, 14, 6] & 156.9 |  [2, 14, 3, 6]\\
\hline
[2, 10, 14] & 90.0 | [2, 10, 14] & 153.0 | [2, 10, 14, 8]  \\
\hline
\end{tabular}
\caption{Table of the amount of times each city is used as a hub in every combination}
\label{amount of times hubs are used in collection} 
\end{table}

The combination of cities 6, 8 and 14 will probably have one of the lower possible costs for this problem for as well the collection cost as the total cost. Using this combination it shows that the cost can be reduced to around 145 thousand euros. 


\subsection{Transfer between hubs}
Transport to the same node is free if the node is a hub. To optimize the total cost an algorithm is made that will be looking at the node with the most flow of parcels to itself and make a hub of this node. After calculating the cost, the next hub with the most flow of parcels to itself will be become a hub. If the solution becomes cheaper by adding a hub, the node is added to a list of hubs. If the total cost increases, nothing happens and the next node (most parcels to itself) becomes a hub. If there are two nodes with the same amount of parcels to itself the algorithm will choose the node with the most incoming parcels. Each node will be connected to the hub with the least cost to transport it there. For example if it cost 11 dollars to transport from node 11 to node 14 and 27 dollars to transport it to node 2, node 11 will be connected with hub 14. This process is repeated until the algorithm goes through all nodes. To illustrate we will use the Large data assignment.

\begin{table}[h]
\begin{tabular}{|p{3cm}|p{3cm}|p{3cm}|}
\hline
nodes that are hubs & flow of parcels to itself (last node) & Total cost\\
\hline
[14] &  181 & 182725 \\
\hline
[14,2]   & 39 & 163586 \\
\hline
[14,2,15] & 26 & 169342 \\
\hline
[14,2,15,6]  & 18 & 156535 \\
\hline
[14,2,15,6,13] & 18 & 161521\\
\hline
[14,2,15,6,13,4] & 11 & 167037   \\
\hline
[14,2,15,6,13,4,3] & 9 & 177126   \\
\hline
[14,2,15,6,13,4,3,9] & 9 & 187710    \\
\end{tabular}
\end{table}


The cheapest situation here is where there are 4 hubs in nodes 2,6,14 and 15. We connect each node with the hub where the cost to transport a parcel is the lowest, the total cost in this situation would be 156 thousand dollars. One can easily see that it is very profitable to make a hub out of node 14, as it is the node with the most incoming parcels (610). In the following subsections we will be looking at the results of the other algorithms and combine those to see if there are similarities.
    
\newpage
    
    
    \subsection{Result of the combined algorithms}
After combining and analysing the results of the different algorithms, a hub will be opened at nodes 14 and 6 as these nodes were also hubs in the cheapest situation of all different algorithms that were used. By opening a hub at node 14 the cost for collecting , transferring and distributing can be reduced, as it is the node with the most amount of incoming parcels. The total cost in this situation where we only have two hubs is 144870. Our intuitive algorithm will then go through the nodes one by one and check which combination (with node 14 and 6) is the cheapest, this node will then be added to the list of hubs. This process is repeated until we go through all nodes. 

\begin{table}[h]
\begin{tabular}{||c |c||}
\hline
Nodes that are hubs & Total cost\\
\hline
[6,14] &  144870 \\
\hline
[6,8,14]   & 144532\\
\hline
[2,6,8,14] & 148189 \\
\hline
[2,5,6,8,14]  & 152593  \\
\hline
[2,5,6,8,12,14] & 160386 \\
\end{tabular}
\end{table}

The most advantageous situation that was found with the intuitive algorithm is 
where we make 3 hubs out of nodes 6,8 and 14. By doing this the total cost is reduced to 144532 dollars which is 338 dollars cheaper than the previous one.    (heb dit snel gemaakt verander het als je wilt)



 
\newpage  
\section{The ILP solver}
We are given a set N of nodes where we can locate a hub. The cost of opening a hub at node i $\in$ N is $F_i$ and the cost of transporting a parcel to node j from node i is $c_{i_j}$, i $\in$ N, j $\in$ N. We need to decide on which nodes we shall open a hub, and from which hub the nodes will be connected. The goal is to minimize the total cost. 
    \subsection{ILP implementation}
    \subsection{IlP result}



\newpage
\section{Expanded algorithm}
The package delivery problem is set out into a data set where only the cost per package and the amount of packages send and received are given. To make this problem more realistic a situation is made where all the packages are collected and distributed using a bus. In this particular case, a bus can hold up to and including a total of one hundred packages. If more packages are needed to move every package, the delivery company will have to send two or more busses. 

This scenario is translated into the algorithm by letting all factors of moving packages 

\newpage
\section{Conclusion}


\section{Appendix} 
\end{document}
