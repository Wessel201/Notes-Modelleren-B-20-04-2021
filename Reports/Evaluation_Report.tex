\documentclass{article}
\usepackage[utf8]{inputenc}

\title{Evaluation Report Modelleren-B}
\author{Bart Stolk \and Dyami Silvério \and Wessel van Sommeren \and Severijn Wiechers}
\date{03 June 2021}

\begin{document}

\maketitle
\newpage
\tableofcontents
\newpage

\section{Introduction}

In this project we were asked to optimize the cost of a parcel delivery network for Post NL using a data set provided. The team consisted of Wessel van Sommeren, Dyami Silvério, Bart Stolk and Severijn Wiechers. We first focused on making an intuitive algorithm to solve the problem. WE did this by splitting up the problem into sub problems and optimizing those. Once we found a good solution for the problem we moved to a Integer Linear Programming implementation of the problem. The code was written in a way that everyone helped to design the algorithm and then one person would write the code. If there were any bugs or complications the rest of the team would help to solve them. The last part of the project consisted of expanding the problem. Once we agreed on how to expand the problem we would start the cycle again, with first looking at a intuitive approach followed by and ILP implementation. 





2. Planning: What was the initial planning? How was it adapted? Did you follow
the planning? What did you do differently? Was the planning useful? Looking
back, what could you have planned better?
\section{Planning}

The initial plan was to meet with each other every Tuesday and Thursday. On Tuesdays we would discuss and work on ideas for the project and on Thursday work was divided between the four of us. This process went well and everyone kept to the agreements. One some days one of us wouldn't be able to show up to the agreed time due to for instance a doctor's appointment, however this was reported a day before to the whole group. The planning generally went well, as everything was completed in time.

\section{Project management}

    The planning and managing of the group and project was done at the end of meetings. Plans were made for what had to be done by next time or some other deadline. On a number of ocassions one or more of the group members were absent, but they either already know what to do for next time or the planning was communicated to them through group chats/the minutes. Some deadlines had to be moved around over the course of the project due to certain parts not always being done in time. However, this never was a cause for great concern, as eventually everything was completed in time.

4. File management and communication: How did you store and manage the files?
How did you communicate? Did problems occur? What could have been done
better?

\section{File management}

    Over the course of the writing of the report we used two main sources for storing data. Firstly, all data sets, Python scripts and report were stored in github. Since most of the group was not familiar with this program, the first day was used to explore git. In hindsight, this was quite useful since github gives many opportunities when working in a group. Furthermore, all the reports in git were linked to overleaf, the second source for data storage. Using \LaTeX as a text editor we were able to give a structured and clean end report. 
    
    Every piece of data was divided into a couple categories. Hereby we differentiated between minutes, reports and Python files. The latter divided into files from the ILP and the intuitive algorithm. By sorting all files into different subgroups we were able to create organized maps such that everything we have written down could easily be retrieved. 

\section{Quality}

    There was no rigid measures by which we tested quality. However, most products made during the course of the project were still tested relatively thoroughly. The cost function was checked for quality multiple times, discovering a number of issues. For example, at one point the function gave different results for the same hubs in a different order. The ILP results were always checked to see if they made sense and compared to the results of the cost function to see if they were better at all. Lastly, reports were checked by varying team members at varying stage of writing them, taking out spelling and style errors where necessary. 

\section{Evaluations}
6. Evaluations of each others work: Every group member again fills out the table
with five criteria for each of the group members. Looking back, were these five
criteria well chosen?
    \begin{itemize}
       \item Punctuality
       \item Motivation / Participation
        \item Adaptability
        \item Quality of the work 
        \item Communication
    \end{itemize}

\begin{table}[]
    \centering
    \begin{tabular}{|c|c|c|c|c|}
        \hline
        - & Bart & Dyami & Wessel & Severijn \\
        \hline
        Punctuality & & & & \\
        \hline
        Motivation
    \end{tabular}
\end{table}

\section{Personal experiences}
7. Personal experiences: Every group member writes approximately half a page on
his/her experiences in the project: What are your experiences in working in this
group? What have you learned? What did you like, and what not? Was it like
you expected, or not? What could you have done better? What do you want to
do differently next time?
\subsection{Bart Stolk}


\subsection{Dyami Silvério}

\subsection{Wessel van Sommeren}

\subsection{Severijn Wiechers}

8. Conclusions (for the whole group).
\section{Conclusions}



\end{document}