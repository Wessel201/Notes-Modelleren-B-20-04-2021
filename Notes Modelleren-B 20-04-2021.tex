\documentclass{article}
\usepackage[utf8]{inputenc}

\title{Eerste opzet van afspraken week 4.1}
\author{Modelleren-B groep 15}
\date{April 2021}

\begin{document}

\maketitle

\section*{Member list}

\begin{itemize}
    \item Severijn Wiechers 4948262
    \item Bart Stolk 5414946
    \item Dyami C.M. Silvério 5172497
    \item Wessel Van Sommeren 5007712
\end{itemize}


\section*{Meeting schedule} 

Aan het eind van iedere meeting wordt een schema opgesteld van de onderdelen die we als groep af willen hebben voor het begin van de volgende meeting. De meetings zijn in principe op de ingeroosterde uren, tenzij anders overeengekomen in een voorgaande meeting. 


\section*{Rough planning}
\textbf{lijst van alle deadlines}
\begin{itemize}
    \item 23-04-2021: Eerste opzet van afspraken
    \item 12-05-2021: peer review
    \item 03-06-2021: draft report
    \item 10-06-2021: feedback draft report
    \item 15-06-2021: Pitch
    \item 17-06-2021: Research report
    \item 18-06-2021: Evaluation report 
\end{itemize}
\textbf{Algemene planning}

In de eerste weken(4.2, 4.3, 4,4, 4.6) ligt de focus op het ontwikkelen van de python file en het algoritme. Week 4.7 voornamelijk ingedeeld voor het afronden van de code en het schrijven van het draft report. In de week na het draft report wordt er tijdens de meeting gekeken naar het feedback geven op de andere verslagen. In week 4.9 wordt de feedback verwerkt in het draft report om tot een eindverlag te komen. Verder zullen hier ook de taken van de ptich worden verdeeld.



\section*{Five criteria for peer review}
    \begin{itemize}
       \item Punctuality
       \item Motivation / Participation
        \item Adaptability
        \item Quality of the work 
        \item Communication
    \end{itemize}



\section*{Quality plan}
In het quality plan worden alle algemene afspraken geschreven voor het opleveren en opslaan van het werk.

\begin{itemize}
    \item Het verslag wordt geschreven in latex/overleaf
    \item De vergaderingen zullen plaatvinden in de groepen van teams
    \item Python files worden aan gewerkt via git
    \item Alle bestanden worden opgedeeld en opgeslagen per onderwerp (denk hierbij aan notulen, verslag of bijvoorbeeld .py files)
    \item De naamgeving van bestanden wordt als volgt gegeven: "Hoofd onderwerp - datum - titel van bestand - maker"

\end{itemize}
\end{document}

